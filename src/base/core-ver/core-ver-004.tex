% output=pdftex 

\setupcolors[state=start]

\starttext

Basic typefile.
\typefile{test-004.tex}

File numbering.
\typefile[numbering=file]{test-004.tex}

Line numbering.
\typefile[numbering=line]{test-004.tex}

Pretty printing (option=TEX).
\typefile[option=TEX]{test-004.tex}

Line numbering, pretty printing (TEX).
\typefile[numbering=line,option=TEX]{test-004.tex}

Line numbering, start=3, pretty printing (TEX).
\typefile[numbering=line,start=3,option=TEX]{test-004.tex}

Line numbering, continue, pretty printing (TEX).
\typefile[numbering=line,continue,option=TEX]{test-004.tex}

Line numbering, start=32, step=2, pretty printing (TEX).
\typefile[numbering=line,start=32,step=2,option=TEX]{test-004.tex}

File numbering, start, step, continue options do nothing.
\typefile[numbering=file,start=3,step=2,option=TEX]{test-004.tex}
\typefile[numbering=file,continue=yes,option=TEX]{test-004.tex}

Put the line numbers in the text (location=intext).
\setuplinenumbering[location=intext]
\typefile[numbering=file,continue=yes,option=TEX]{test-004.tex}

File numbering, color=blue, option=TEX.
\typefile[numbering=file,color=blue,option=TEX]{test-004.tex}

\stoptext
