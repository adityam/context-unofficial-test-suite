\starttext

% \setuplinenumbering
  % [\c!conversion=\v!numbers,
   % \c!start=1,
   % \c!step=1,
   % \c!method=\v!first,
   % \c!continue=v!no,
   % \c!location=\v!left,
   % \c!style=,
   % \c!color=,
   % \c!width=2em,
   % \c!left=,
   % \c!right=,
   % \c!command=,
   % \c!distance=\zeropoint,
   % \c!align=\v!auto]

% \setupnotes
  % [\c!location=\v!page,
   % \c!way=\v!by\v!part,
  % %\c!conversion=,
   % \c!rule=\v!on,
   % \c!before=\blank,
   % \c!bodyfont=\v!small,
  % %\c!style=,
  % %\c!color=,
  % %\c!after=,
  % %\c!rulecolor=,
   % \c!rulethickness=\linewidth,
   % \c!frame=\v!off,
   % \c!margindistance=.5em,
   % \c!columndistance=1em,
   % \c!distance=.125em,
   % \c!align=\v!normal,
   % \c!tolerance=\v!tolerant,
   % \c!split=\v!tolerant,
  % %\c!width=\makeupwidth,
  % %\c!width=\ifdim\hsize<\makeupwidth\hsize\else\makeupwidth\fi,
   % \c!width=\defaultnotewidth,
   % \c!height=\textheight,
   % \c!numbercommand=\high,
   % \c!command=\noteparameter\c!numbercommand, % downward compatible
   % \c!separator=,% \@@koseparator,
   % \c!textcommand=\high,
   % \c!textstyle=\tx,
  % %\c!textcolor=,
   % \c!interaction=\v!yes,
  % %\c!factor=,
  % %\c!scope=, % \v!text \v!page
% \c!prefixconnector=.,
% \c!prefix=\v!no,
   % \c!next=\autoinsertnextspace, % new, experimental with startnotes
   % \c!n=1]

\setuppapersize[S6][S6]
\setuplayout[width=middle,height=middle,margin=1.5cm,footer=0pt,header=1cm]
\setupcolors[state=start]
\setuptyping[option=color]

\showframe

\title{\type{continue=yes|no}, \type{rule=on|off}}

\subject{Issues: None}

\page

\startbuffer[test]
\setuplinenumbering[continue=no]
\setuplinenote [linenote] [rule=off,frame=on,framecolor=darkred]
\startlinenumbering
\subject{Let's start}
test test test \dorecurse{40}{test }.
\linenote {A simple linenote does not have a number range}
\startlinenote [one] {A linenote environment has a range that covers the
first line of an environment up to the last.}
\dorecurse{40}{test }.
\stoplinenote [one]
\linenote {A simple linenote does not have a number range}
\dorecurse{30}{test }.
\par
\stoplinenumbering
\subject{Continue or not?}
\startlinenumbering
\dorecurse{40}{test }.
\stoplinenumbering
\stopbuffer

{\typebuffer[test] \page \getbuffer[test] \page}

\startbuffer[test]
\setuplinenumbering[continue=yes]
\setuplinenote [linenote] [rule=on,frame=on,framecolor=darkred,n=2]
\startlinenumbering [101]
\subject{Let's start}
test test test \dorecurse{40}{test }.
\linenote {A simple linenote does not have a number range}
\startlinenote [one] {A linenote environment has a range that covers the
first line of an environment up to the last.}
\dorecurse{40}{test }.
\stoplinenote [one]
\linenote {A simple linenote does not have a number range}
\dorecurse{30}{test }.
\par
\stoplinenumbering
\subject{Continue or not?}
\startlinenumbering
\dorecurse{40}{test }.
\stoplinenumbering
\stopbuffer

{\typebuffer[test] \page \getbuffer[test] \page}
\stoptext
