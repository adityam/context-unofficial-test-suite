\starttext

\definelinenote   [linenote]
\definelinenote   [extranote]
\definelinenote   [supernote] [extranote]

\setupnote        [linenote] [frame=on,framecolor=darkred]
\setupnote        [extranote][frame=on,framecolor=darkgreen]

\setupnote        [linenote] [paragraph=yes,numbercommand=,inbetween=\hskip.5em\vl\hskip0.2em\vl\hskip.5em]
\setupdescriptions[linenote] [display=no,location=serried,distance=.5em]

\setupnote        [extranote][paragraph=yes,numbercommand=,inbetween=\quad]
\setupdescriptions[extranote][display=yes,location=serried,distance=.5em,style=bold]

\setupnote        [footnote] [paragraph=yes,numbercommand=,inbetween=\hskip.5em\vl\hskip.5em]
\setupdescriptions[footnote] [display=no,location=serried,distance=.5em]

\startlinenumbering

This nation, turning 100 years old, had no {\em Odyssey}, no
St.~George slaying the dragon, no Prometheus. The emerging
American genius for making a lot of money was a\linenote {ln:one}
poor substitute for King Arthur and his knights \footnote {one}
(although the Horatio Alger myth \linenote {ln:two} of rags to
riches was good for a lot of mileage). Without a mythology and set
of ancient heroes to call its own, America had to \extranote
{en:one} manufacture its heroes. So the mythmaking machinery of
nineteenth|-|century American media created a suitable heroic
archetype in the cowboys of the Wild West. The image was of the
undaunted \extranote {en:two} cattle drivers \footnote {two}
living a life of reckless individualism, braving the elements,
staving off brutal Indian \supernote{whow} attacks. Or of heroic
lawmen dueling with six|-|guns in the streets at high noon. This
artificial Wild West became America's Iliad.

\stoplinenumbering

\stoptext
