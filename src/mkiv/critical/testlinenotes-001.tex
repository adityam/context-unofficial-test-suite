\starttext

% \setupnotes
  % [\c!location=\v!page,
   % \c!way=\v!by\v!part,
  % %\c!conversion=,
   % \c!rule=\v!on,
   % \c!before=\blank,
   % \c!bodyfont=\v!small,
  % %\c!style=,
  % %\c!color=,
  % %\c!after=,
  % %\c!rulecolor=,
   % \c!rulethickness=\linewidth,
   % \c!frame=\v!off,
   % \c!margindistance=.5em,
   % \c!columndistance=1em,
   % \c!distance=.125em,
   % \c!align=\v!normal,
   % \c!tolerance=\v!tolerant,
   % \c!split=\v!tolerant,
  % %\c!width=\makeupwidth,
  % %\c!width=\ifdim\hsize<\makeupwidth\hsize\else\makeupwidth\fi,
   % \c!width=\defaultnotewidth,
   % \c!height=\textheight,
   % \c!numbercommand=\high,
   % \c!command=\noteparameter\c!numbercommand, % downward compatible
   % \c!separator=,% \@@koseparator,
   % \c!textcommand=\high,
   % \c!textstyle=\tx,
  % %\c!textcolor=,
   % \c!interaction=\v!yes,
  % %\c!factor=,
  % %\c!scope=, % \v!text \v!page
% \c!prefixconnector=.,
% \c!prefix=\v!no,
   % \c!next=\autoinsertnextspace, % new, experimental with startnotes
   % \c!n=1]

\setuppapersize[S6][S6]
\setuplayout[width=middle,height=middle,margin=1.5cm,footer=0pt,header=1cm]
\setupcolors[state=start]
\setuptyping[option=color]

\showframe

\title{Some Basics}

\subject{Issues:}

\startitemize
\item \type{\startlinenumbering} conversion from the last-defined buffer
affects earlier buffers. renaming other buffers did not help. {\bf high}
\item related to above: try commenting the last buffer and see what happens
to the earlier ones; a slight different effect. {\bf high}
\item \type{method=first} on page~3 had no affect: no numeral for first line
\item line flushing is incorrect: numerals by default should be flush up
against the margin. Examples:
\startitemize
\item 1--9 on the left of page~2 should be flush up against the margin.
\item line numerals for notes are not properly aligned.
\stopitemize
 {\bf high}
\stopitemize


\page

\startbuffer[test]
\startlinenumbering
test test test
\dorecurse{108}{test }.
\linenote {A simple linenote does not have a number range}
\startlinenote [one] {A linenote environment has a range that covers the
					  first line of an environment up to the last.}
\dorecurse{110}{test }.
\linenote {A simple linenote does not have a number range}
\dorecurse{110}{test }.
\stoplinenote [one]
\par
\stoplinenumbering
\stopbuffer

{\typebuffer[test] \page \getbuffer[test] \page}

\startbuffer[test]
\setuplinenumbering[step=5,method=first,conversion=numbers]
\startlinenumbering
test test test
\dorecurse{108}{test }.
\linenote {A simple linenote does not have a number range}
\startlinenote [one] {A linenote environment has a range that covers the
					  first line of an environment up to the last.}
\dorecurse{110}{test }.
\linenote {A simple linenote does not have a number range}
\dorecurse{110}{test }.
\stoplinenote [one]
\par
\stoplinenumbering
\stopbuffer

{\typebuffer[test] \page \getbuffer[test] \page}

\startbuffer[test]
\setuplinenumbering[start=101,step=5,method=next,conversion=romannumerals]
\startlinenumbering
test test test
\dorecurse{108}{test }.
\linenote {A simple linenote does not have a number range}
\startlinenote [one] {A linenote environment has a range that covers the
					  first line of an environment up to the last.}
\dorecurse{110}{test }.
\linenote {A simple linenote does not have a number range}
\dorecurse{110}{test }.
\stoplinenote [one]
\par
\stoplinenumbering
\stopbuffer

{\typebuffer[test] \page \getbuffer[test] \page}

\startbuffer[test]
\setuplinenumbering[start=101,step=5,method=next,conversion=Romannumerals]
\startlinenumbering
test test test
\dorecurse{108}{test }.
\linenote {A simple linenote does not have a number range}
\startlinenote [one] {A linenote environment has a range that covers the
					  first line of an environment up to the last.}
\dorecurse{110}{test }.
\linenote {A simple linenote does not have a number range}
\dorecurse{110}{test }.
\stoplinenote [one]
\par
\stoplinenumbering
\stopbuffer

{\typebuffer[test] \page \getbuffer[test] \page}

\stoptext
